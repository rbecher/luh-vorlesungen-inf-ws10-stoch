%% Basierend auf einer TeXnicCenter-Vorlage von Tino Weinkauf.
%%%%%%%%%%%%%%%%%%%%%%%%%%%%%%%%%%%%%%%%%%%%%%%%%%%%%%%%%%%%%%

%%%%%%%%%%%%%%%%%%%%%%%%%%%%%%%%%%%%%%%%%%%%%%%%%%%%%%%%%%%%%
%% HEADER
%%%%%%%%%%%%%%%%%%%%%%%%%%%%%%%%%%%%%%%%%%%%%%%%%%%%%%%%%%%%%
\documentclass[a4paper,twoside,12pt]{article}

\usepackage[ngerman]{babel}
\usepackage[utf8]{inputenc}
\usepackage{setspace}
\usepackage{cancel} % strikethrough in math mode
\usepackage[osf]{mathpazo}  % Mathpazo package includes accompanying math fonts

%% Paragraphen sollen nicht eingerueckt sein
\setlength{\parskip}{\smallskipamount}
\setlength{\parindent}{0pt}

\usepackage{lmodern} %Type1-Schriftart f\"ur nicht-englische Texte

%% Packages f\"ur Grafiken & Abbildungen %%%%%%%%%%%%%%%%%%%%%%
\usepackage{graphicx} %%Zum Laden von Grafiken
\usepackage{graphics}
%\usepackage{subfig} %%Teilabbildungen in einer Abbildung
%\usepackage{tikz} %%Vektorgrafiken aus LaTeX heraus erstellen
%\usepackage{qtree}

%% Packages f\"ur Formeln %%%%%%%%%%%%%%%%%%%%%%%%%%%%%%%%%%%%%
\usepackage{amsmath}
\usepackage{amsthm}
\usepackage{amsfonts}

%% Zeilenabstand %%%%%%%%%%%%%%%%%%%%%%%%%%%%%%%%%%%%%%%%%%%%
\usepackage{setspace}
\singlespacing        %% 1-zeilig (Standard)

%% Andere Packages %%%%%%%%%%%%%%%%%%%%%%%%%%%%%%%%%%%%%%%%%%
\usepackage{a4wide} %%Kleinere Seitenr\"ander = mehr Text pro Zeile.
\usepackage{fancyhdr} %%Fancy Kopf- und Fußzeilen
\usepackage{fancybox}
\usepackage{cite}
\usepackage{tocbibind}

\usepackage[hyperindex,colorlinks,bookmarks,urlcolor=blue]{hyperref}

%% FancyHeader Definitionen %%%%%%%%%%%%%%%%a%%%%%%%%%%%%%%%%%%%%%%%%%%

%\lhead{Stchastik A, WS 2009}
%\chead{}
%\rhead{Mitschrift}

%\pagestyle{fancy}{
%   \fancyhead[LO,RE]{\rightmark}
%   \fancyhead[RO,LE]{\leftmark}
%   \fancyfoot[C]{\thepage}
%}


%% Basierend auf einer TeXnicCenter-Vorlage von Tino Weinkauf.

%%%%%%%%%%%%%%%%%%%%%%%%%%%%%%%%%%%%%%%%%%%%%%%%%%%%%%%%%%%%%%


%%%%%%%%%%%%%%%%%%%%%%%%%%%%%%%%%%%%%%%%%%%%%%%%%%%%%%%%%%%%%

%% OPTIONEN

%%%%%%%%%%%%%%%%%%%%%%%%%%%%%%%%%%%%%%%%%%%%%%%%%%%%%%%%%%%%%

%%

%% ACHTUNG: Sie benötigen ein Hauptdokument, um diese Datei

%%          benutzen zu können. Verwenden Sie im Hauptdokument

%%          den Befehl "\input{dateiname}", um diese

%%          Datei einzubinden.

%%



%%%%%%%%%%%%%%%%%%%%%%%%%%%%%%%%%%%%%%%%%%%%%%%%%%%%%%%%%%%%%

%% ABKÜRZUNGEN

%%%%%%%%%%%%%%%%%%%%%%%%%%%%%%%%%%%%%%%%%%%%%%%%%%%%%%%%%%%%%


%%Definitionen für Abbildung-Referenzen

\newcommand{\abb}[1]{(Abbildung \ref{#1})}

\newcommand{\ABB}[1]{Abbildung \ref{#1}}


%%Definitionen für Tabellen-Referenzen

\newcommand{\tab}[1]{(Tabelle \ref{#1})}

\newcommand{\TAB}[1]{Tabelle \ref{#1}}


%%Definitionen für Seiten-Referenzen

\newcommand{\seite}[1]{(Seite \pageref{#1})}

\newcommand{\SEITE}[1]{Seite \pageref{#1}}


%%Definitionen für Code

\newcommand{\precode}[1]{\textbf{\footnotesize #1}}

\newcommand{\code}[1]{\texttt{\footnotesize #1}}



%% Basierend auf einer TeXnicCenter-Vorlage von Tino Weinkauf.

%%%%%%%%%%%%%%%%%%%%%%%%%%%%%%%%%%%%%%%%%%%%%%%%%%%%%%%%%%%%%%


%%%%%%%%%%%%%%%%%%%%%%%%%%%%%%%%%%%%%%%%%%%%%%%%%%%%%%%%%%%%%

%% OPTIONEN

%%%%%%%%%%%%%%%%%%%%%%%%%%%%%%%%%%%%%%%%%%%%%%%%%%%%%%%%%%%%%

%%

%% ACHTUNG: Sie benötigen ein Hauptdokument, um diese Datei

%%          benutzen zu können. Verwenden Sie im Hauptdokument

%%          den Befehl "\input{dateiname}", um diese

%%          Datei einzubinden.

%%


%%%%%%%%%%%%%%%%%%%%%%%%%%%%%%%%%%%%%%%%%%%%%%%%%%%%%%%%%%%%%

%% OPTIONEN FÜR ABSTÄNDE

%%%%%%%%%%%%%%%%%%%%%%%%%%%%%%%%%%%%%%%%%%%%%%%%%%%%%%%%%%%%%


%%Abstand zwischen den Absätzen: halbe Höhe vom kleinen x

\setlength{\parskip}{0.5ex}


%%Einzug am Anfang eines Absatzes: auf Null setzen

%\setlength{\parindent}{0ex}


%%Zeilenabstand: 1.5 fach

%% ==> Erwägen Sie, anstelle dieses Kommandos das Paket 'setspace' zu verwenden.

%\linespread{1.5}



%%%%%%%%%%%%%%%%%%%%%%%%%%%%%%%%%%%%%%%%%%%%%%%%%%%%%%%%%%%%%

%% OPTIONEN FÜR KOPF- UND FUSSZEILEN

%%%%%%%%%%%%%%%%%%%%%%%%%%%%%%%%%%%%%%%%%%%%%%%%%%%%%%%%%%%%%

%%Beispiel für recht nette Kopf- und Fußzeilen

%% ==> Nutzen Sie '\usepackage{fancyhdr}' und '\pagestyle{fancy}'

%% ==> im Hauptdokument, um diese zu benutzen.

%\pagestyle{fancy}

\renewcommand{\chaptermark}[1]{\markboth{#1}{}}

\renewcommand{\sectionmark}[1]{\markright{\thesection\ #1}}

\fancyhf{}

\fancyhead[LE,RO]{\thepage}

\fancyhead[LO]{\rightmark}

\fancyhead[RE]{\leftmark}

\fancypagestyle{plain}{%

    \fancyhead{}

    \renewcommand{\headrulewidth}{0pt}

}




%% Basierend auf einer TeXnicCenter-Vorlage von Tino Weinkauf.

%%%%%%%%%%%%%%%%%%%%%%%%%%%%%%%%%%%%%%%%%%%%%%%%%%%%%%%%%%%%%%


%%%%%%%%%%%%%%%%%%%%%%%%%%%%%%%%%%%%%%%%%%%%%%%%%%%%%%%%%%%%%

%% PDF-Informationen

%%%%%%%%%%%%%%%%%%%%%%%%%%%%%%%%%%%%%%%%%%%%%%%%%%%%%%%%%%%%%

%%

%% ACHTUNG: Sie benötigen ein Hauptdokument, um diese Datei

%%          benutzen zu können. Verwenden Sie im Hauptdokument

%%          den Befehl "\input{dateiname}", um diese

%%          Datei einzubinden.

%%


\pdfinfo{                               % Zusatzinformationen in PDF-Datei;

                                        % alle Werte sind optional.

    /Author (Ronald Becher)

   % /CreationDate (D:20100928120000)    % Datum der Erstellung

                                        % (D:JJJJMMTThhmmss)

                                        % JJJJ  Jahr

                                        % MM    Monat

                                        % TT    Tag

                                        % hh    Stunden

                                        % mm    Minuten

                                        % ss    Sekunden

                                        %

                                        % Standard: Das aktuelle Datum

                                        %

    /ModDate (\date)         % Datum der letzten Modifikation

    /Creator (TeX && TXC)               % Standard: "TeX"

    /Producer (pdfTeX)                  % Standard: "pdfTeX" + pdftex version

    /Title (Vorlesungsmitschrift Stochastik Wintersemester 2010)

    /Subject (Vorlesungsmitschrift Stochastik Wintersemester 2010)

    /Keywords (stochastik, mathematik, vorlesung, mitschrift)

}



\title{Stochastik Wintersemester 2009\\Leibniz Universit\"at Hannover\\Vorlesungsmitschrift}

\author{Dozent: \href{mailto:rgrubel@stochastik.uni-hannover.de}{Prof. Dr. R. Gr\"ubel}\\ \vspace{1em} \\
Mitschrift von\\
\href{mailto:rb@ronald-becher.com}{Ronald Becher}\\
\href{mailto:eugen@eugenkiss.com}{Evgenij Kiss}}

\begin{document}
\pagestyle{empty}
\maketitle
\begin{abstract}
    Diese Mitschrift wird erstellt im Zuge der Vorlesung \href{http://www.stochastik.uni-hannover.de/ws2010.html}{Stochastik A} im Wintersemester 2010 an der \href{http://www.uni-hannover.de}{Leibniz Universit\"at Hannover}. Obwohl mit gro\ss er Sorgfalt geschrieben, werden sich sicherlich Fehler einschleichen. Diese bitte ich zu melden, damit sie korrigiert werden k\"onnen. Ihr k\"onnt euch dazu an jeden der genannten Autoren (mit Ausnahme des Dozenten) wenden.

    Dieses Skript wird prim\"ar \"uber \href{http://github.com/rbecher/luh-vorlesungen-inf-ws10-stoch}{Github} verteilt und "`gepflegt"'. Dort kann man es auch forken, verbessern und dann (idealerweise) ein "`Pull Request"' losschicken. Siehe auch  \href{http://github.com/guides}{Github Guides}. Auch wenn ihr gute Grafiken zur Verdeutlichung beitragen k\"onnt, d\"urft ihr diese gerne schicken oder selbst einf\"ugen (am besten als \LaTeX -geeignete Datei (Tikz, SVG, PNG, \dots)).

    \textbf{Hinweis}: Es wird o.B.d.A. davon abgeraten, die Vorlesung zu vers\"aumen, nur weil eine Mitschrift angefertigt wird! Selbiges gilt f\"ur den \"Ubungsbetrieb! Stochastik besteht ihr nur, wenn ihr auch zu der Veranstaltung geht!
\end{abstract}

\newpage % \newpage works far better than \pagebreak

\tableofcontents

%Part: F\"ur große Themenbl\"ocke (z.B. Einf\"uhrung, Kombinatorik, etc)
%Section: F\"ur Abschnitte (z.B. Mengen, von Mengen zu anderen Mengen, etc)
%Subsection: Beweise, S\"atze, Lemmata

\newpage

\pagestyle{fancy}
\addtolength{\headheight}{\baselineskip}
\renewcommand{\sectionmark}[1]{\markboth{#1}{}}
\renewcommand{\subsectionmark}[1]{\markright{#1}}
%\rhead{\leftmark\\\rightmark}
\fancyhead[LE,RO]{Mitschrift Stochastik A, WS 2010}
\fancyhead[LO,RE]{\leftmark\\\rightmark}
% Here comes content

%\part{Einf\"uhrung}

%\section{Inhalte}

%Dies ist ein Paragraph

%\newpage

\part{Ein mathematisches Modell für Zufallsexperimente}

Man fasst die möglichen Ergebnisse $\omega$ zu einer Menge $\Omega$ zusammen; man nennt $\Omega$ den \textit{Ergebnisraum} oder auch \textit{Stichprobenraum}. \textit{Ereignisse} werden durch Teilmengen von $\Omega$ beschrieben.
Eine \textit{Aussage} über das Ergebnis wird zu der Menge $A$ aller $\omega \in \Omega$, für diese die Aussage richtig ist.

\section{Beispiel - Würfelwurf}

\begin{displaymath}
\Omega = \{1,2,3,4,5,6\}
\end{displaymath}

Das Ereignis ``Es kommt eine gerade Zahl heraus'' wird beschrieben durch

\begin{displaymath}
A = \{2,4,6\}
\end{displaymath}

Das Ereignis ``Es kommt eine 6 heraus'' wird (in Mengenschreibweise) beschrieben durch $A = \{6\}$.
Solche Ereignisse, die nur aus einem Element bestehen, nennt man \textit{Elementarereignisse}.
Aussagen über das Ergebnis können \textit{logisch kombiniert} werden. Auf der Ereignisseite (also in Teilmengen von $\Omega$) wird dies zu Mengenoperationen. Werden zwei Ereignisse $A, B$ durch Mengen repräsentiert, so ist

\begin{eqnarray*}
A^c \text{ das Ereignis ``$A$ tritt nicht ein''}\\
A \cap B \text{ das Ereignis "'Beide Ereignisse können eintreten"'}\\
A \cup B \text{ eines der Ereignisse tritt ein}
\ldots
\end{eqnarray*}

Hier kommt eine Grafik über eine Matrix .... oder ein Foto der Tafel
A = Pasch\\
B = Augensumme kleiner 5\\
A geschnitten B

Man kann dies mit mehr als zwei Ereignissen machen.
Wenn $A,B,C$ Ereignisse sind, wodurch wird ''genau eins davon tritt ein'' beschrieben?

\begin{equation}
A \cap B^c \cap C^c + A^c \cap  B \cap C^c + A^c \cap B^c \cap C
\end{equation}
% TODO entweder die Eqn doppeln oder auf das + hinweisen wegen disjunkter Mengen

\section{Was ist Wahrscheinlichkeit?}

Unsere zentrale Frage lautet nun: ``Was ist Wahrscheinlichkeit?''\\
Der Mathematiker wählt ein Axiomensystem (vgl. Geometrie/Raum).
Ausgangspunkt hierfür ist der Alltagsbegriff.
Was bedeutet es also, wenn beim Wurf einer fairen Münze mit Wahrscheinlichkeit $\frac{1}{2}$ das Ergebnis ``Kopf'' kommt?\\

\subsection{Frequentistische Auffassung}

Bezeichnet $N_n(A)$ die Anzahl der Versuche bei $n$ Wiederholungen des Experiments, bei dem $A$ kommt, so sollte die \textbf{relative Häufigkeit} $\frac{1}{n} N_n(A)$ bei großem $n$ in der Nähe der Wahrscheinlichkeit $p$ des Ereignisss $A$ liegen.

\subsection{Subjektivistische Auffassung}

$p$ drückt den ``Grad meines Glaubens'' an das Eintreten von $A$ auf einer Skala von $0$ (``kein Glaube'') bis $1$ (``Gewissheit'') aus.
Kann über Wetten formalisiert werden.

\noindent \textbf{Anmerkung:} Diese beiden Auffassungen sind nicht disjunkt!

Für relative Häufigkeiten gilt
\begin{eqnarray*}
0 \leq \frac{1}{n} N_n(A) \leq 1 \\
\frac{1}{n} N_n(\Omega) = 1 \\
\frac{1}{n} N_n(\emptyset) = 0\\
\frac{1}{n} N_n(A+B) = \frac{1}{n} N_n(A) + \frac{1}{n} N_n(B)
\end{eqnarray*}

\section{Definition: Kolmogorov-Axiome}

Gegeben seien ein (nicht leerer) Ergebnisraum $\Omega$ und ein System $\mathcal{A}$ von Teilmengen von $\Omega$, das System der Ereignisse (bei endlichen oder abzählbar unendlichen $\Omega$ ist $\mathcal{A}$ einfach die Potenzmenge $\mathcal{P}(\Omega)$, also die Menge \textbf{aller} Teilmengen von $\Omega$).
Eine \textbf{Wahrscheinlichkeit} oder besser ein \textbf{Wahrscheinlichkeitsmaß} (WMaß) ist eine Abbildung $P:\mathcal{A} \rightarrow \mathbb{R}$ mit folgenden Eigenschaften:
\begin{description}
 \item[(A1)] $P(A)\geq 0\ \forall A \in \mathcal{A},\ P(\Omega)=1$ ($P$ steht für ``probability'')
 \item[(A2)] Für alle paarweise disjunkten Ereignisse $A_1,A_2,A_3,\ldots$ gilt
 \begin{displaymath}
	P(\bigcup_{i=1}^\infty A_i) = P(\sum_{i=1}^\infty A_i) = \sum_{i=1}^\infty P(A_i)
 \end{displaymath}
 Dies nennt man ``$\sigma$-Additivität''.
\end{description}

\section{Wahrscheinlichkeitsraum}
Das Tripel $(\Omega,\mathcal{A},P)$ nennt man einen \textbf{Wahrscheinlichkeitsraum}.

\noindent Es folgen nun erste Folgerungen aus den Axiomen.

\section{Satz}
% TODO: Enumerate soll mit kleinen Buchstaben arbeiten
\renewcommand{\labelitemi}{\alph}
\begin{enumerate}
	\item $P(\emptyset) = 0$
	\item $P(A) \leq 1 \forall A \in \mathcal{A}$
	\item $P(A^c) = 1 - P(A) \forall A \in \mathcal{A}$ (Übergang zum \textbf{Gegenereignis})
	\item $A \subset B \Rightarrow P(A) \leq P(B)$ (Monotonie)
	\item $P(A_1 + A_2 + \ldots + A_n) = P(A_1) + P(A_2) + \ldots + P(A_n)$ für paarweise disjunkte $A_1,\ldots,A_n \in \mathcal{A}$ (endliche Additivität)
	\item Boolesche Ungleichung: $P(A_1 \cup \ldots \cup A_n) \leq P(A_1) + \ldots + P(A_n)$
	\item Die \textbf{Siebformel}
	\[
		\mathsf{P}\left(\bigcup_{i=1}^nA_i\right) = \sum_{k=1}^n(-1)^{k+1}\!\!\sum_{I\subseteq\{1,\dots,n\},\atop |I|=k}\!\!\!\!\mathsf{P}\left(\bigcap_{i\in I}A_i\right)
	\]
\end{enumerate}

\subsection{Beweis}
\begin{enumerate}
	\item Mit (A2) und $A_j=\emptyset \forall j$ erhält man $P(\emptyset) = \sum_{i=1}^\infty P(\emptyset)$. 
	Wegen $P(\emptyset) \in \mathbb{R}$ bedeutet dies $P(\emptyset) = 0$.
	\item Verwende (A2) mit $A_j=\emptyset$ für $j>k$ und dann Teil (a):
	\begin{eqnarray*}
		P(A_1 \cup \ldots \cup A_k) = \\
		P(A_1 \cup \ldots \cup A_k \cup \emptyset \cup \emptyset \ldots) = \\
		P(A_1) + \ldots + P(A_k) + P(\emptyset) + \ldots = \\
		P(A_1) + \ldots + P(A_k)
	\end{eqnarray*}
	\item $1 = P(\Omega) = P(A + A^c) = P(A) + P(A^c)$
	\item $B = A + B \cap A^c$\\
	$P(B) = P(A) + (P(B\cap A^c) \geq P(A)$
	\item bei $n=2$: $P(A \cup B) \neq P(A) + P(B)$
\end{enumerate}


\textbf{Hinweis:} Jetzt seid ihr so weit in dieser Mitschrift gekommen, bitte gleicht diese mit eurer handschriftlichen Mitschrift ab und sendet mir eure Verbesserungen. Gerne auch gleich in \LaTeX\ ;-)
\part{Laplace-Experimente, elementare Kombinatorik}

Unser Modell für ein Zufallsexperiment ist ein sogenannter \emph{WRaum}
$(\Omega,\mathcal{A},\mathcal{P})$, wobei $\Omega$ die möglichen
\emph{Ergebnisse} des Experiments enthält, $\mathcal{A}$ ein System von
Teilmengen von $\Omega$ ist, die \emph{Ereignisse} (für uns in der Regel die
Potenzmenge $\mathcal{P}(\Omega)$), und $P$ eine Funktion, die jedem Ereignis
$A$ eine Wahrscheinlichkeit $P(A)$ zuordnet.

Spezialfall Laplace-Experimente:

\begin{displaymath}
\#\Omega < \infty, \ \mathcal{A} = \mathcal{P}(\Omega), \ P(A) =
\frac{\#A}{\#\Omega}
\end{displaymath}

(vielleicht schon bekannt: Wahrscheinlichkeit als Anzahl der günstigen
Ergebnisse durch die Anzahl der möglichen Ergebnisse)

Ergibt sich aus bzw. ist äquivalent zu der Aussage, dass alle
Elementarereignisse dieselbe Wahrscheinlichkeit haben. Typische Anwendung:
Werfen eines symmetrischen Gegenstandes (Würfel, Münze, etc.)


\subsection{Beispiel: Würfelwurf}

Ein Würfel wird zweimal geworfen. Mit welcher Wahrscheinlichkeit erhält man
Augensumme 7 bzw. 6?

Wir betrachten das Laplace-Experiment über
\begin{align*}
\Omega &= \{ (i,j) : 1 \leq i,h \leq 6 \} = \{ 1,\ldots,6 \} \times
\{ 1,\ldots,6 \} \\
\#\Omega &= 36 \\
A_7 &= \{ (i,j) \in \Omega : i+j = 7\} = \{ (1,6), (2,5), (3,4), (4,3), (5,2),
(6,1) \} \\
P(A_7) &= \frac{6}{36} = \frac{1}{6} \\
A_6 &= \{ (1,5), (2,4), (3,3), (4,2), (5,1) \} \\
P(A_6) &= \frac{5}{36} ( < P(A_7))
\end{align*}


\section{Die Kunst des Zählens}

Bei Laplace-Experimenten läuft die Bestimmung von Wahrscheinlichkeiten also
auf die ``Kunst des Zählens'' (elementare Kombinatorik) hinaus. Zwei zentrale
Regeln:

\begin{enumerate}
\item Gibt es eine bijektive Abbildung von $A$ nach $B$, so gilt $\#A = \#B$
\item Sind $A$ und $B$ disjunkt, so gilt $\#(A \cup B) = \#A + \#B$
\end{enumerate}

Regel 2 lässt sich auf mehr als zwei Mengen verallgemeinern: Sind
$A_1,\ldots,A_n$ disjunkt, so gilt $\#(A_1 + \ldots + A_n) = \#A_1 + \#A_2 +
\ldots + \#A_n$.

Wichtige Folgerung: Im Falle $C \subset A \times B$ mit $\#\{ y : (x,y) \in C
\} = k$ fuer alle $x \in A$ gilt $\#C = k \cdot \#A$.


\subsection{Beispiel: Hörsaal}

Hörsaal mit $n$ Reihen mit jeweils $m$ Plätzen. In jeder Reihe sind $k$
Plätze besetzt. Man hat insgesamt $n \cdot k$ Zuhörer.

Spezialfall: Bei $C = A \times B$ erhält man $\#(A \times B) = \#A \cdot \#B$.


\section{Vier Standardfamilien}

Permutationen/Kombinationen mit/ohne Wiederholung.

\subsection{Permutationen mit Wiederholung}

Wieviele Moeglichkeiten gibt es $m$ Objekte (Kugeln) auf $n$ Plätze (Urnen)
zu verteilen? Man hat $n$ Möglichkeiten für das 1. Objekt, wieder $n$ fuer
das 2. etc., also insgesamt $n^m$ Möglichkeiten. Formal:

\[ \{ (i_1,\ldots,i_m) : i_j \in \{ 1,\ldots,n \} \text{ für } j = 1,\ldots,m
\} = n^m \]

Dabei bedeutet das Tupel $(i_1,\ldots,i_m)$, dass Objekt $j$ in Urne $i_j$
gelegt wird. Allgemeiner: Bei endlichen Mengen $A$ und $B$ bezeichnet $B^A$
die Menge aller Funktionen $f: A \to B$, also: $\#(B^A) = (\#B)^{\#A}$.


\subsection{Permutationen ohne Wiederholung}

Was passiert, wenn man nur \emph{injektive} Funktionen zulässt? Klar: Man
braucht $\#B \geq \#A$. Dies setzen wir voraus. Man hat $n$ Möglichkeiten
für das erste Objekt. Ist dieses ausgeteilt, so bleiben $n-1$ Möglichkeiten
fuer das zweite Objekt (Regel 2). Sind die ersten beiden ausgeteilt, so
bleiben $n-2$ für das dritte Objekt etc., man hat also

\[ n \cdot (n-1) \cdot (n-2) \cdot (n-3) \cdot \ldots \cdot (n-m+1) =
\frac{n!}{(n-m)!} = (n)_m \]

Möglichkeiten.
Spezialfall: $A=B$. Dann ist jedes injektive $f$ automatisch surjektiv, also
bijektiv. Allgemein:

\begin{align*}
\#\{ f \in B^A : f \text{ injektiv} \} &= \frac{(\#B)!}{(\#B - \#A)!} \\
A=B : \#\{ \underbrace{f: A \to A : f \text { bijektiv}}_{ \substack{\text {
\scriptsize Die Elemente dieser Menge nennt}\\ \text{\scriptsize man \emph
{Permutationen} von $A$ }}} \} &= (\#A)!
\end{align*}

Standardanwendung: Kartenmischen. Bei $32$ Karten gibt es $32!$ Möglichkeiten
fuer das Mischergebnis.


\subsection{Kombination ohne Wiederholung}

Formal: Wir führen auf $B^A$ eine Äquivalenzrelation ein:

\[ f \sim g :\Leftrightarrow \exists \pi: A \to A \text{ bijektiv}: f = g \circ
\pi \]
\begin{alignat*}{3}
(&f \sim f  \text{ ?}&\quad
&\text{wegen } f = f \circ \pi \text{ mit } \pi = id \\
&f \sim f \Rightarrow g \sim f \text{ ? }&
&f = g \circ \pi \text{ impliziert } g = f \circ \pi^{-1} \\
&f \sim g, g \sim f \Rightarrow f \sim h \text{ ?}&
&f = g \circ \pi, g = h \circ \sigma \Rightarrow f = h \circ \sigma \circ \pi)
\end{alignat*}

Führt auf Zerlegung von $\{ f \in B^A : f \text{ injektiv}\}$ in
Äquivalenzklassen. Wie viele gibt es? Alle haben dieselbe Anzahl von
Elementen, nämlich $m!$ (bei $\#A = m$).


\subsubsection{Beispiel: Lotto}

Beim Lotto 6 aus 49 erhält man zunächst ein Tupel $(i_1,\ldots,i_6)$ mit $i_j
\in \{1,\ldots,49\}$ und $i_j \neq i_k$ fuer $j \neq k$ (eine injektive
Abbildung von $A = \{1,\ldots,6\}$ in $B = \{1,\ldots,49\}$). Dies wird
aufsteigend angeordnet:

\[ k_1,\ldots,k_6 \text{ mit } k_1 < \ldots < k_6 \]

Es gibt $6!$ Möglichkeiten. Formal gilt bei injektiven Funktionen $f : f
\circ \pi = f \circ \sigma \Rightarrow \pi = \sigma$, d.h. die Anzahl der
Elemente einer Äquivalenzklasse ist gleich der Anzahl der Permutationen
(bij: Selbstabbildungen) der Menge $A$, also $(\#A)!$. Also:

\[ \#\{ f \in B^A : f \text{ injektiv} \} = \frac{(\#B)!}{(\#B - \#A)!} \]

Jeweils $(\#A)!$ Elemente werden zu einer Äquivalenzklasse zusammengefasst,
d.h. die gesuchte Anzahl ist

\[ \frac{(\#B)!}{(\#B - \#A)! \cdot (\#A)!} = {\#B \choose \#A} \]

Alternativ:

\[ \{ (i_1,\ldots,i_m): 1 \leq i_1 < i_2 < \ldots < i_m \leq n \} = {n \choose
m} \]

Spezialfall: Wieviele Teilmengen vom Umfang $m$ hat eine Menge von $n$
Elementen? Antwort: $n \choose m$.


\subsection{Kombinationen mit Wiederholung}

\[ \{(i_1,\ldots,i_m): 1 \leq i_1 \leq i_2 \leq \ldots \leq i_m \leq n \} =
{n+m-1 \choose m} \]

\textbf{Beweis mit Regel 1:} Wir können die Menge der $m$-Kombinationen aus
$n$ \emph{mit} Widerholung bijektiv abbilden auf die Menge der
$m$-Kombinationen aus $n+m-1$ \emph{ohne} Wiederholung durch

\[ (i_1,\ldots,i_m) \mapsto (i_1, i_2+1, i_3+2, \ldots, i_m + m-1) \]


\subsection{Zusammenfassende Tabelle}

\begin{center}
\begin{tabular}{l|c|c}
    ~         & Permutationen       & Kombinationen     \\
    \cline{2-3}
    mit Wdh.  & $n^m$               & $n+m-1 \choose m$ \\
    ohne Wdh. & $\frac{n!}{(n-m)!}$ & $n \choose m$     \\
\end{tabular}
\end{center}

{\scriptsize Grundmengen $A,B$ mit $\#A=n$, $\#B=m$.}



\section{Kombinatorik-Beispiele}

\subsection{Beispiel: 10-facher Münzwurf}

Mit welcher W. kommt exakt 7-mal ``Kopf''?
$0$ sei Wappen, $1$ sei Kopf.

Das Modell ist offenbar ein Laplace-Experiment über $\{0,1\}^10$, also ist 
\[\Omega=\{0,1\}^10 = \{ (i_1,\ldots,i_10) : i_j \in \{0,1\}, j = 1,\ldots,10 \}\]

$(1,1,1,1,1,0,0,0,0,0)$ ist das Ereignis/Ergebnis ``Die ersten 5 Würfe ergeben ``Kopf'', alle späteren ``Wappen'' ''

\noindent \textbf{Allgemein gilt}: 
$\omega = (\i_1,\ldots,i_10)$ heißt, dass im $j$-ten Versuch das mit $i_j$ kodierte Ergebnis kommt.\\
\textbf{Nächster Schritt}:
Identifiziere das interessierende Ereignis als Teilmenge $A \subset \Omega$:
\[ A = \{ \omega=(i_1,\ldots,i_10) \in \Omega : i_1+\ldots + i_10 = 7 \} \]
\textbf{In Worten}:
Alle $10$-Tupel, die an $7$ Positionen eine $1$ und an $3$ Positionen eine $0$ haben.

\noindent Wie immer bei Laplace-Experimenten gilt: 
\[P(A) = \frac{|A|}{|\Omega|} \]
Klar ist: $|\Omega| = 2^{10} = 1024$ ($\Omega$ besteht aus den $10$-Permutationen mit Wiederholung aus einer Menge von $2$ Elementen).

\noindent Die nächste Frage heißt: 
Wie viele Elemente hat $A$?\\
Die $7$ $1$en müssen auf die $10$ möglichen Positionen (ohne Wiederholung) verteiltwerden.
$7$-Kombination ohne Wiederholung mit $10$ ELementen:
\[ 10 \choose 7 = \frac{10!}{7!3!} = \frac{10\cdot 9 \cdot 6}{3 \cdot 2} = 5 \cdot 3 \cdot 8 = 120 \]
Also \[ P(A) = \frac{120}{1024} = \frac{15}{128} \]

\subsection{Beispiel: Geburtstagsproblem}

Wie groß ist die W. dafür, dass von $n$ Personen (mindestens) zwei am gleichen Tag Geburtstag haben?
Vernünftige Einschränkungen an $n$: $n \geq 2, n \leq 365$ (Wir ignorieren Schaltjahre, Zwillinge, etc.)

Wie gehen wieder von einem Laplace-Experiment aus, und zwar über
\[ \Omega = \{ (i_1,\ldots,i_n) : i_j = \{ 1,\ldots,365 \} \} \]
Dabei bedeutet $i_j=t$, dass die $j$-te Person am $k$-ten Tag des Jahres Geburtstag hat.
$|\Omega| = 365^n$ ($n$ Permutationen mit Wiederholung aus einer Menge mit $365$ Elementen.

\textbf{Nächster Schritt} (wieder):
Identifiziere $A$ als Teilmenge von $\Omega$.
\[ A = \{ (i_1,\ldots,i_n) \in \Omega : \exists j_k=\{1,\ldots,n\}\text{ mit }j\neq k \cap i_j=i_k  \} \]

Hier nützlich: Übergang zum Komplementär-Ereignis
\[ A^c = \{ (i_1,\ldots,i_n) \in \Omega : i_j \neq i_k,\ j \neq k \}\]

$A^c$ ist also die Menge der $n$-Permutationen aus einer Menge von $365$ Elementen, nun also \textbf{ohne} Wiederholung:
$365 \cdot 364 \cdot \ldots \cdot (365 - n +1)$
Also
\[ P(A) = 1-P(A^c) = 1 - \frac{365 \cdot 364 \cdot \ldots \cdot (365 - n +1)}{365^n} \]

Man sieht an der Formel, dass $P(A)$ mitwachsenden $n$ größer wird.
Interessanterweise erhält man ab $n=23$ einen Wert $P(A) \geq \frac{1}{2}$.

\subsection{Beispiel: Wichteln}

$n$ Kinder verpacken jeweils ein Geschenk.
Diese werden zufällig an die Kinder verteilt: jedes Kind erhält also exakt ein Geschenk.

Wie groß ist nun die Wahrscheinlichkeit, dass dies ``ohne Tränen funktioniert'', so dass kein Kind sein eigenes Geschenk erhält?
$\Omega$ sei die Menge aller $n$-Permutationen ohne Wiederholung aus $n$ Interpretationen.
$\omega = (i_1,\ldots,i_n)$ mit $i_j=k$ bedeutet, dass dsa $j$-te Kind Geschenk mit der Nummer $k$ erhält.
Also $|\Omega| = n!$

$A$ ist nun die Menge der fixpunktfreien Permutationen. Also ist $A^c$ die Menge aller Permutationen, die mindestens einen Fixpunkt haben.

Wir suchen
\[ B_j = \{ (i_1,\ldots,i_n)\in \Omega:\ i_j=j \} \]
($j$ ist ein Fixpunkt, Kind Nummer $j$ erhält sein eigenes Geschenk zurück)

Klar: \[ A^c = \bigcup_{j=1}^n B_j \]

Nebenrechnung: $|B_j = (n-1)!$

Schade: Die $B_j$ sind nicht disjunkt.


\[ A^c = \{ (i_1,\ldots,i_n) \in \Omega:\ \exists j \in \{1,\ldots, n\} \text{ mit } i_j=j \]

Typisches Anwendungsgebiet fürdie Siebformel: % TODO Href dorthin
Für $n=2$ gilt also z.B.
\[ P(B_1 \cup B_2) = P(B_1) + P(B_2) - P(B_1\cap B_2) \]

Bei $|H| = k$ besteht $\bigcap_{j\in H}B_j$ aus allen Permutationen, deren Werte in den $k$ Positionen aus $H$ festgelegt sind, die übrigen $n-k$ sind frei.

Für alle $H$ mit $|H|=k$ erhält man also $P\left(\bigcap_{j\in H} B_j\right) = \frac{(n-k)!}{n!}$.
Hängt also nur von $|H|$ ab.

Wie vele $H \subset \{ 1,\ldots,n \}$ mit $|H|=k$ gibt es?

$k$-Kombinationen ohne Wdhl. aus $|H|$ ist $n \choose k$

Damit 
\[P(A) = 1- P(A^c) = 1-P\left(\bigcup_{j=1}^n B_j\right) = 1-\sum_{k=1}^n(-1)^{k+1} \cdot {n \choose k} \cdot \frac{(n-k)!}{n!} = \sum_{k=0}^n \frac{(-1)^k}{k!}\] % TODO Sieht seltsam aus ...

Aus der Analysis ist die Reihendarstellung der Experimentalfunktion bekannt
\[e^x = \sum_{k=0}^\infty \frac{x^k}{k!}\]

Mit $n\rightarrow\infty$ geht die gesuchte Wahrscheinlichkeit also gegen $e^{-1} = \frac{1}{e} \approx 0.3679\ldots$

\subsection{Beispiel: Kartenspiel}

Ein Kartenspiel mit $52$ Karten wird gut gemischt, die oberen $5$ werden umgedreht.
Mit welcher Wahrscheinlichkeit erhält man Ereignis $A$, einen ``Flush'' (alle haben die gleichen Farben), bzw. Ereignis $B$, $4$ Karten mit der gleichen Wertigkeit?

\[P(A) = 4 \cdot \frac{{13 \choose 5}}{{52 \choose 5}} \] (Möglichkeiten für die Farbe mal die Möglichkeit, aus der Farbe 5 auszusuchen durch Gesamtmenge)

\[ P(B) = \frac{13 \cdot 48}{{52 \choose 5}} \]
(Auswahl des Kartenwerts mal Möglichkeit für die 5. karte(???) durch Gesamtmenge)

\[ \frac{P(A)}{P(B)} = \frac{33}{4} \]

Also ....

\textbf{Hinweis:} Jetzt seid ihr so weit in dieser Mitschrift gekommen, bitte gleicht diese mit eurer handschriftlichen Mitschrift ab und sendet mir eure Verbesserungen. Gerne auch gleich in \LaTeX\ ;-)

%\appendix

%\part{Anhang}

%\renewcommand\refname{Literaturangaben}

%\nocite{*}

%\bibliography{books}

%\bibliographystyle{output}

\end{document}
