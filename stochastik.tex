%% Basierend auf einer TeXnicCenter-Vorlage von Tino Weinkauf.
%%%%%%%%%%%%%%%%%%%%%%%%%%%%%%%%%%%%%%%%%%%%%%%%%%%%%%%%%%%%%%

%%%%%%%%%%%%%%%%%%%%%%%%%%%%%%%%%%%%%%%%%%%%%%%%%%%%%%%%%%%%%
%% HEADER
%%%%%%%%%%%%%%%%%%%%%%%%%%%%%%%%%%%%%%%%%%%%%%%%%%%%%%%%%%%%%
\documentclass[a4paper,twoside,12pt]{article}                

\usepackage[ngerman]{babel}
\usepackage[utf8]{inputenc}
\usepackage{setspace}
%\usepackage{cancel} % strikethrough in math mode
\usepackage[osf]{mathpazo}	% Mathpazo package includes accompanying math fonts

\usepackage{lmodern} %Type1-Schriftart f\"ur nicht-englische Texte
             
%% Packages f\"ur Grafiken & Abbildungen %%%%%%%%%%%%%%%%%%%%%%
\usepackage{graphicx} %%Zum Laden von Grafiken
\usepackage{graphics}
%\usepackage{subfig} %%Teilabbildungen in einer Abbildung
%\usepackage{tikz} %%Vektorgrafiken aus LaTeX heraus erstellen
%\usepackage{qtree}

%% Packages f\"ur Formeln %%%%%%%%%%%%%%%%%%%%%%%%%%%%%%%%%%%%%
\usepackage{amsmath}
\usepackage{amsthm}
\usepackage{amsfonts}

%% Zeilenabstand %%%%%%%%%%%%%%%%%%%%%%%%%%%%%%%%%%%%%%%%%%%%
\usepackage{setspace}
\singlespacing        %% 1-zeilig (Standard)

%% Andere Packages %%%%%%%%%%%%%%%%%%%%%%%%%%%%%%%%%%%%%%%%%%
\usepackage{a4wide} %%Kleinere Seitenr\"ander = mehr Text pro Zeile.
\usepackage{fancyhdr} %%Fancy Kopf- und Fußzeilen
\usepackage{fancybox}
\usepackage{cite}
\usepackage{tocbibind}

\usepackage[hyperindex,colorlinks,bookmarks,urlcolor=blue]{hyperref}    

%% FancyHeader Definitionen %%%%%%%%%%%%%%%%a%%%%%%%%%%%%%%%%%%%%%%%%%%

%\lhead{Stchastik A, WS 2009}
%\chead{}
%\rhead{Mitschrift}

%\pagestyle{fancy}{
%	\fancyhead[LO,RE]{\rightmark}
%	\fancyhead[RO,LE]{\leftmark}
%	\fancyfoot[C]{\thepage}
%}                                


%% Basierend auf einer TeXnicCenter-Vorlage von Tino Weinkauf.

%%%%%%%%%%%%%%%%%%%%%%%%%%%%%%%%%%%%%%%%%%%%%%%%%%%%%%%%%%%%%%


%%%%%%%%%%%%%%%%%%%%%%%%%%%%%%%%%%%%%%%%%%%%%%%%%%%%%%%%%%%%%

%% OPTIONEN

%%%%%%%%%%%%%%%%%%%%%%%%%%%%%%%%%%%%%%%%%%%%%%%%%%%%%%%%%%%%%

%%

%% ACHTUNG: Sie benötigen ein Hauptdokument, um diese Datei

%%          benutzen zu können. Verwenden Sie im Hauptdokument

%%          den Befehl "\input{dateiname}", um diese

%%          Datei einzubinden.

%%



%%%%%%%%%%%%%%%%%%%%%%%%%%%%%%%%%%%%%%%%%%%%%%%%%%%%%%%%%%%%%

%% ABKÜRZUNGEN

%%%%%%%%%%%%%%%%%%%%%%%%%%%%%%%%%%%%%%%%%%%%%%%%%%%%%%%%%%%%%


%%Definitionen für Abbildung-Referenzen

\newcommand{\abb}[1]{(Abbildung \ref{#1})}

\newcommand{\ABB}[1]{Abbildung \ref{#1}}


%%Definitionen für Tabellen-Referenzen

\newcommand{\tab}[1]{(Tabelle \ref{#1})}

\newcommand{\TAB}[1]{Tabelle \ref{#1}}


%%Definitionen für Seiten-Referenzen

\newcommand{\seite}[1]{(Seite \pageref{#1})}

\newcommand{\SEITE}[1]{Seite \pageref{#1}}


%%Definitionen für Code

\newcommand{\precode}[1]{\textbf{\footnotesize #1}}

\newcommand{\code}[1]{\texttt{\footnotesize #1}}



%% Basierend auf einer TeXnicCenter-Vorlage von Tino Weinkauf.

%%%%%%%%%%%%%%%%%%%%%%%%%%%%%%%%%%%%%%%%%%%%%%%%%%%%%%%%%%%%%%


%%%%%%%%%%%%%%%%%%%%%%%%%%%%%%%%%%%%%%%%%%%%%%%%%%%%%%%%%%%%%

%% OPTIONEN

%%%%%%%%%%%%%%%%%%%%%%%%%%%%%%%%%%%%%%%%%%%%%%%%%%%%%%%%%%%%%

%%

%% ACHTUNG: Sie benötigen ein Hauptdokument, um diese Datei

%%          benutzen zu können. Verwenden Sie im Hauptdokument

%%          den Befehl "\input{dateiname}", um diese

%%          Datei einzubinden.

%%


%%%%%%%%%%%%%%%%%%%%%%%%%%%%%%%%%%%%%%%%%%%%%%%%%%%%%%%%%%%%%

%% OPTIONEN FÜR ABSTÄNDE

%%%%%%%%%%%%%%%%%%%%%%%%%%%%%%%%%%%%%%%%%%%%%%%%%%%%%%%%%%%%%


%%Abstand zwischen den Absätzen: halbe Höhe vom kleinen x

\setlength{\parskip}{0.5ex}


%%Einzug am Anfang eines Absatzes: auf Null setzen

%\setlength{\parindent}{0ex}


%%Zeilenabstand: 1.5 fach

%% ==> Erwägen Sie, anstelle dieses Kommandos das Paket 'setspace' zu verwenden.

%\linespread{1.5}



%%%%%%%%%%%%%%%%%%%%%%%%%%%%%%%%%%%%%%%%%%%%%%%%%%%%%%%%%%%%%

%% OPTIONEN FÜR KOPF- UND FUSSZEILEN

%%%%%%%%%%%%%%%%%%%%%%%%%%%%%%%%%%%%%%%%%%%%%%%%%%%%%%%%%%%%%

%%Beispiel für recht nette Kopf- und Fußzeilen

%% ==> Nutzen Sie '\usepackage{fancyhdr}' und '\pagestyle{fancy}'

%% ==> im Hauptdokument, um diese zu benutzen.

%\pagestyle{fancy}

\renewcommand{\chaptermark}[1]{\markboth{#1}{}}

\renewcommand{\sectionmark}[1]{\markright{\thesection\ #1}}

\fancyhf{}

\fancyhead[LE,RO]{\thepage}

\fancyhead[LO]{\rightmark}

\fancyhead[RE]{\leftmark}

\fancypagestyle{plain}{%

    \fancyhead{}

    \renewcommand{\headrulewidth}{0pt}

}




%% Basierend auf einer TeXnicCenter-Vorlage von Tino Weinkauf.

%%%%%%%%%%%%%%%%%%%%%%%%%%%%%%%%%%%%%%%%%%%%%%%%%%%%%%%%%%%%%%


%%%%%%%%%%%%%%%%%%%%%%%%%%%%%%%%%%%%%%%%%%%%%%%%%%%%%%%%%%%%%

%% PDF-Informationen

%%%%%%%%%%%%%%%%%%%%%%%%%%%%%%%%%%%%%%%%%%%%%%%%%%%%%%%%%%%%%

%%

%% ACHTUNG: Sie benötigen ein Hauptdokument, um diese Datei

%%          benutzen zu können. Verwenden Sie im Hauptdokument

%%          den Befehl "\input{dateiname}", um diese

%%          Datei einzubinden.

%%


\pdfinfo{                               % Zusatzinformationen in PDF-Datei;

                                        % alle Werte sind optional.

    /Author (Ronald Becher)

   % /CreationDate (D:20100928120000)    % Datum der Erstellung

                                        % (D:JJJJMMTThhmmss)

                                        % JJJJ  Jahr

                                        % MM    Monat

                                        % TT    Tag

                                        % hh    Stunden

                                        % mm    Minuten

                                        % ss    Sekunden

                                        %

                                        % Standard: Das aktuelle Datum

                                        %

    /ModDate (\date)         % Datum der letzten Modifikation

    /Creator (TeX && TXC)               % Standard: "TeX"

    /Producer (pdfTeX)                  % Standard: "pdfTeX" + pdftex version

    /Title (Vorlesungsmitschrift Stochastik Wintersemester 2010)

    /Subject (Vorlesungsmitschrift Stochastik Wintersemester 2010)

    /Keywords (stochastik, mathematik, vorlesung, mitschrift)

}


                                      
\title{Stochastik Wintersemester 2009\\Leibniz Universit\"at Hannover\\Vorlesungsmitschrift}

\author{Dozent: \href{mailto:rgrubel@stochastik.uni-hannover.de}{Prof. Dr. R. Gr\"ubel}\\ \vspace{1em} \\
Mitschrift von\\
\href{mailto:rb@ronald-becher.com}{Ronald Becher}\\
\href{mailto:eugen@eugenkiss.com}{Evgenij Kiss}}

\begin{document}
\pagestyle{empty}
\maketitle
\begin{abstract}
	Diese Mitschrift wird erstellt im Zuge der Vorlesung \href{http://www.stochastik.uni-hannover.de/ws2010.html}{Stochastik A} im Wintersemester 2010 an der \href{http://www.uni-hannover.de}{Leibniz Universit\"at Hannover}. Obwohl mit gro\ss er Sorgfalt geschrieben, werden sich sicherlich Fehler einschleichen. Diese bitte ich zu melden, damit sie korrigiert werden k\"onnen. Ihr k\"onnt euch dazu an jeden der genannten Autoren (mit Ausnahme des Dozenten) wenden.
	
	Dieses Skript wird prim\"ar \"uber \href{http://github.com/rbecher/luh-vorlesungen-inf-ws10-stoch}{Github} verteilt und "`gepflegt"'. Dort kann man es auch forken, verbessern und dann (idealerweise) ein "`Pull Request"' losschicken. Siehe auch  \href{http://github.com/guides}{Github Guides}. Auch wenn ihr gute Grafiken zur Verdeutlichung beitragen k\"onnt, d\"urft ihr diese gerne schicken oder selbst einf\"ugen (am besten als \LaTeX -geeignete Datei (Tikz, SVG, PNG, \dots)).
	
	\textbf{Hinweis}: Es wird o.B.d.A. davon abgeraten, die Vorlesung zu vers\"aumen, nur weil eine Mitschrift angefertigt wird! Selbiges gilt f\"ur den \"ubungsbetrieb! Stochastik besteht ihr nur, wenn ihr auch zu der Veranstaltung geht!
\end{abstract}

\newpage % \newpage works far better than \pagebreak

\tableofcontents

%Part: F\"ur große Themenbl\"ocke (z.B. Einf\"uhrung, Kombinatorik, etc)
%Section: F\"ur Abschnitte (z.B. Mengen, von Mengen zu anderen Mengen, etc)
%Subsection: Beweise, S\"atze, Lemmata

\newpage

\pagestyle{fancy}
\addtolength{\headheight}{\baselineskip}
\renewcommand{\sectionmark}[1]{\markboth{#1}{}}
\renewcommand{\subsectionmark}[1]{\markright{#1}}
%\rhead{\leftmark\\\rightmark}
\fancyhead[LE,RO]{Mitschrift Stochastik A, WS 2010}
\fancyhead[LO,RE]{\leftmark\\\rightmark}
% Here comes content

%\part{Einf\"uhrung}

%\section{Inhalte}

%Dies ist ein Paragraph

%\newpage

\part{Ein mathematisches Modell für Zufallsexperimente}

Man fasst die möglichen Ergebnisse $\omega$ zu einer Menge $\Omega$ zusammen; man nennt $\Omega$ den \textit{Ergebnisraum} oder auch \textit{Stichprobenraum}. \textit{Ereignisse} werden durch Teilmengen von $\Omega$ beschrieben.
Eine \textit{Aussage} über das Ergebnis wird zu der Menge $A$ aller $\omega \in \Omega$, für diese die Aussage richtig ist.

\section{Beispiel - Würfelwurf}

\begin{displaymath}
\Omega = \{1,2,3,4,5,6\}
\end{displaymath}

Das Ereignis ``Es kommt eine gerade Zahl heraus'' wird beschrieben durch

\begin{displaymath}
A = \{2,4,6\}
\end{displaymath}

Das Ereignis ``Es kommt eine 6 heraus'' wird (in Mengenschreibweise) beschrieben durch $A = \{6\}$.
Solche Ereignisse, die nur aus einem Element bestehen, nennt man \textit{Elementarereignisse}.
Aussagen über das Ergebnis können \textit{logisch kombiniert} werden. Auf der Ereignisseite (also in Teilmengen von $\Omega$) wird dies zu Mengenoperationen. Werden zwei Ereignisse $A, B$ durch Mengen repräsentiert, so ist

\begin{eqnarray*}
A^c \text{ das Ereignis ``$A$ tritt nicht ein''}\\
A \cap B \text{ das Ereignis "'Beide Ereignisse können eintreten"'}\\
A \cup B \text{ eines der Ereignisse tritt ein}
\ldots
\end{eqnarray*}

Hier kommt eine Grafik über eine Matrix .... oder ein Foto der Tafel
A = Pasch\\
B = Augensumme kleiner 5\\
A geschnitten B

Man kann dies mit mehr als zwei Ereignissen machen.
Wenn $A,B,C$ Ereignisse sind, wodurch wird ''genau eins davon tritt ein'' beschrieben?

\begin{equation}
A \cap B^c \cap C^c + A^c \cap  B \cap C^c + A^c \cap B^c \cap C
\end{equation}
% TODO entweder die Eqn doppeln oder auf das + hinweisen wegen disjunkter Mengen

\section{Was ist Wahrscheinlichkeit?}

Unsere zentrale Frage lautet nun: ``Was ist Wahrscheinlichkeit?''\\
Der Mathematiker wählt ein Axiomensystem (vgl. Geometrie/Raum).
Ausgangspunkt hierfür ist der Alltagsbegriff.
Was bedeutet es also, wenn beim Wurf einer fairen Münze mit Wahrscheinlichkeit $\frac{1}{2}$ das Ergebnis ``Kopf'' kommt?\\

\subsection{Frequentistische Auffassung}

Bezeichnet $N_n(A)$ die Anzahl der Versuche bei $n$ Wiederholungen des Experiments, bei dem $A$ kommt, so sollte die \textbf{relative Häufigkeit} $\frac{1}{n} N_n(A)$ bei großem $n$ in der Nähe der Wahrscheinlichkeit $p$ des Ereignisss $A$ liegen.

\subsection{Subjektivistische Auffassung}

$p$ drückt den ``Grad meines Glaubens'' an das Eintreten von $A$ auf einer Skala von $0$ (``kein Glaube'') bis $1$ (``Gewissheit'') aus.
Kann über Wetten formalisiert werden.

\noindent \textbf{Anmerkung:} Diese beiden Auffassungen sind nicht disjunkt!

Für relative Häufigkeiten gilt
\begin{eqnarray*}
0 \leq \frac{1}{n} N_n(A) \leq 1 \\
\frac{1}{n} N_n(\Omega) = 1 \\
\frac{1}{n} N_n(\emptyset) = 0\\
\frac{1}{n} N_n(A+B) = \frac{1}{n} N_n(A) + \frac{1}{n} N_n(B)
\end{eqnarray*}

\section{Definition: Kolmogorov-Axiome}

Gegeben seien ein (nicht leerer) Ergebnisraum $\Omega$ und ein System $\mathcal{A}$ von Teilmengen von $\Omega$, das System der Ereignisse (bei endlichen oder abzählbar unendlichen $\Omega$ ist $\mathcal{A}$ einfach die Potenzmenge $\mathcal{P}(\Omega)$, also die Menge \textbf{aller} Teilmengen von $\Omega$).
Eine \textbf{Wahrscheinlichkeit} oder besser ein \textbf{Wahrscheinlichkeitsmaß} (WMaß) ist eine Abbildung $P:\mathcal{A} \rightarrow \mathbb{R}$ mit folgenden Eigenschaften:
\begin{description}
 \item[(A1)] $P(A)\geq 0\ \forall A \in \mathcal{A},\ P(\Omega)=1$ ($P$ steht für ``probability'')
 \item[(A2)] Für alle paarweise disjunkten Ereignisse $A_1,A_2,A_3,\ldots$ gilt
 \begin{displaymath}
	P(\bigcup_{i=1}^\infty A_i) = P(\sum_{i=1}^\infty A_i) = \sum_{i=1}^\infty P(A_i)
 \end{displaymath}
 Dies nennt man ``$\sigma$-Additivität''.
\end{description}

\section{Wahrscheinlichkeitsraum}
Das Tripel $(\Omega,\mathcal{A},P)$ nennt man einen \textbf{Wahrscheinlichkeitsraum}.

\noindent Es folgen nun erste Folgerungen aus den Axiomen.

\section{Satz}
% TODO: Enumerate soll mit kleinen Buchstaben arbeiten
\renewcommand{\labelitemi}{\alph}
\begin{enumerate}
	\item $P(\emptyset) = 0$
	\item $P(A) \leq 1 \forall A \in \mathcal{A}$
	\item $P(A^c) = 1 - P(A) \forall A \in \mathcal{A}$ (Übergang zum \textbf{Gegenereignis})
	\item $A \subset B \Rightarrow P(A) \leq P(B)$ (Monotonie)
	\item $P(A_1 + A_2 + \ldots + A_n) = P(A_1) + P(A_2) + \ldots + P(A_n)$ für paarweise disjunkte $A_1,\ldots,A_n \in \mathcal{A}$ (endliche Additivität)
	\item Boolesche Ungleichung: $P(A_1 \cup \ldots \cup A_n) \leq P(A_1) + \ldots + P(A_n)$
	\item Die \textbf{Siebformel}
	\[
		\mathsf{P}\left(\bigcup_{i=1}^nA_i\right) = \sum_{k=1}^n(-1)^{k+1}\!\!\sum_{I\subseteq\{1,\dots,n\},\atop |I|=k}\!\!\!\!\mathsf{P}\left(\bigcap_{i\in I}A_i\right)
	\]
\end{enumerate}

\subsection{Beweis}
\begin{enumerate}
	\item Mit (A2) und $A_j=\emptyset \forall j$ erhält man $P(\emptyset) = \sum_{i=1}^\infty P(\emptyset)$. 
	Wegen $P(\emptyset) \in \mathbb{R}$ bedeutet dies $P(\emptyset) = 0$.
	\item Verwende (A2) mit $A_j=\emptyset$ für $j>k$ und dann Teil (a):
	\begin{eqnarray*}
		P(A_1 \cup \ldots \cup A_k) = \\
		P(A_1 \cup \ldots \cup A_k \cup \emptyset \cup \emptyset \ldots) = \\
		P(A_1) + \ldots + P(A_k) + P(\emptyset) + \ldots = \\
		P(A_1) + \ldots + P(A_k)
	\end{eqnarray*}
	\item $1 = P(\Omega) = P(A + A^c) = P(A) + P(A^c)$
	\item $B = A + B \cap A^c$\\
	$P(B) = P(A) + (P(B\cap A^c) \geq P(A)$
	\item bei $n=2$: $P(A \cup B) \neq P(A) + P(B)$
\end{enumerate}


\textbf{Hinweis:} Jetzt seid ihr so weit in dieser Mitschrift gekommen, bitte gleicht diese mit eurer handschriftlichen Mitschrift ab und sendet mir eure Verbesserungen. Gerne auch gleich in \LaTeX\ ;-)                 

%\appendix

%\part{Anhang}

%\renewcommand\refname{Literaturangaben}

%\nocite{*}

%\bibliography{books}

%\bibliographystyle{output}

\end{document}              