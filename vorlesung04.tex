\part{Bedingte Wahrscheinlichkeiten und Unabhängigkeit}

\section{Motivation}
Was ist die Wahrscheinlichkeit für einen Pasch (= zwei gleiche Ergebnisse) beim Wurf zweier Würfel.
Wie ändert sich diese Wahrscheinlichkeit, wenn man als Zusatzinformation die Augensumme hat?

\textbf{Erinnerung}:
In der frequentistischen Sichtweise haben Wahrscheinlichkeiten etwas mit relativen Häufigkeiten zu tun.
Bei $n$ Wiederholungen sei $N_n(A)$ die Anzahl der Versuche, in denen $A$ eintritt, $ \frac{1}{n}N_n(A) $ ist die relative Häufigkeit von $A$.

Es liegt nahe, die \textbf{bedingte} Wahrscheinlichkeit ``von $B$ gegeben $A$'' in Bezug zu bringen mit
\[
	\frac{N_n(A\cap B)}{N_n(A)} = \frac{\frac{1}{n}N_n(A\cap B)}{\frac{1}{n}N_n(A)}
\]

\section{Definition: Bedingte Wahrscheinlichkeit}

Ist $A$ ein Ereignis mit $P(A)>0$, so heißt
\begin{equation}
	P(B|A) := \frac{P(A \cap B)}{P(A)}
\end{equation}
die \textbf{bedingte Wahrscheinlichkeit von B unter A}.

\subsection{Beispiel bei Augensumme 6}

A: ``Die Augensumme ist 6''

B: ``Es ist ein Pasch''

\[
	P(B|A) = \frac{P(A \cap B)}{P(A)} = \frac{ \frac{1}{36} }{ \frac{5}{36} } = \frac{1}{5}
\]

\section{Rechenregeln}

Es folgen drei einfache Rechenregeln zum Umgang mit bedingten Wahrscheinlichkeiten

\begin{description}
	\item[Die Multiplikationsregel] Sind $A_1,\ldots,A_n$ Ereignisse mit $P(A_1\cap\ldots\cap A_n)>0$, so gilt
		\begin{equation}
			P(A_1\cap\ldots\cap A_n) = P(A_1)\cdot P(A_2|A_1) \cdot P(A_3|A_1 \cap A_2) \cdot \ldots \cdot P(A_n|A_1\cap\ldots\cap A_{n-1})
		\end{equation}
		In Worten: Die W. eines Durchschnitts lässt sich als Produkt von bedingten W. schreiben.\\
		Der Beweis ist einfach und es gab ihn in der Vorlesung. Wer mag einen Link raussuchen?
	\item[Gesetz von der totalen W.] Es sei $A_1,\ldots,A_n$ eine \emph{Ereignispartition}, d.h. % TODO Bild
		$A_1,\ldots,A_n$ paarweise disjunkt, die Vereinigung ist ganz $\Omega$. Dann gilt für jedes Ereignis $B$
		\begin{equation}
			P(B) = \sum_{i=1}^n P(B|A_i)\cdot P(A_i)
		\end{equation}
	\item[Formel von Bayes] $A_1,\ldots,A_n,B$ wie bei der totalen W., zusätzlich $P(B)>0$. Dann gilt
		\begin{equation}
			\label{def:bayes}
			P(A_i|B) = \frac{P(A_i)\cdot P(B|A_i)}{\sum_{k=1}^n P(A_k)\cdot P(B|A_k)}
		\end{equation}
		Interpretation beim Nenner: Im Falle $P(A_k)=0$ sei $P(A_k)\cdot P(B|A_k) = 0$; die entsprechenden Summanden werden weggelassen.
\end{description}

\subsection{Beispiel}

Ein bestimmter medizinischer Test ist zu $95\%$ effektiv beim Erkennen einer bestimmten Krankheit, liefert allerdings bei $1\%$ der gesunden Personen einen ``falschen Alarm''!
Angenommen, $0,5\%$ der Bevölkerung haben diese Krankheit.
Mit welcher W. hat jemand die Krankheit, wenn der Test dies behauptet (``der Test ist positiv'')?

Wie stets bei Textaufgaben müssen diese Informationen zunächst in Formeln transformiert werden. Wir setzen
\begin{align*}
    A &: \text{``Person hat die Krankheit''} \\
    B &: \text{``Test ist positiv''}
\end{align*}

Damit gilt
\begin{alignat*}{4}
    &P(B|A)   && =  0.95  \\
    &P(B|A^c) && =  0.01  \\
    &P(A)     && =  0.005
\end{alignat*}
Es geht um $P(A|B)$.

Wir wenden die Bayes-Formel \eqref{def:bayes} an mit der Partition $A,A^c$.
\begin{eqnarray*}
	P(A|B)	&= \frac{P(A)\cdot P(B|A)}{P(A)\cdot P(B|A) + P(A^c)\cdot P(B|A^c)} \\
			&= \frac{0.005 \cdot 0.95}{0.005 \cdot 0.95 + 0.995 \cdot 0.01} \\
			&\approx 0.323
\end{eqnarray*}
Also: Die W. dafür, dass man die Krankheit hat, wenn der Test dies anzeigt, ist \textbf{nur} etwa ein Drittel.\\
Als Buchempfehlung an dieser Stelle sei ``Der Hund, der Eier legt'' empfohlen. % TODO link zu amazon

\section{Unabhängigkeit}

Ein zentraler Begriff der Stochastik ist die \textbf{Unabhängigkeit}.
Informell: Wenn die Zusatzinformation, dass $A$ eintritt, nichts bringt, dann heißen $A$ und $B$ voneinander unabhängig.
Um Probleme mit $P(A)=0$ zu vermeiden, helfe folgende Definition

Die Ereignisse $A,B$ heißen \textbf{unabhängig}, wenn gilt
\begin{equation}
	P(A \cap B) = P(A) \cdot P(B)
\end{equation}

Bei mehr als zwei Ereignissen muss man aufpassen:\\
Eine Familie $A_i,\ i\in I$ heißt unabhängig, wenn für jede endliche Teilmenge $J \subset I$ die Produktregel gilt:
\begin{equation}
	P\left( \bigcap_{i\in J}A_i\right) = \prod_{i\in J} P(A_i)
\end{equation}

Bei $I=\{1,2,3\}$ muss also gelten:
\begin{align*}
	P(A_1 \cap A_2) &= P(A_1)\cdot P(A_2) \\
	P(A_1 \cap A_3) &= P(A_1)\cdot P(A_3) \\
	P(A_2 \cap A_3) &= P(A_2)\cdot P(A_3) \\
	P(A_1 \cap A_2 \cap A_3) &= P(A_1)\cdot P(A_2)\cdot P(A_3)
\end{align*}

\subsection{Beispiel: Funktionieren von Netzwerken}

Wir gehen von $5$ Zwischen-Komponenten (Knoten) aus, die unabhängig voneinander mit W. $p$ funktionieren.\\
Hier sollte ein Bild hin ;-)\\

Mit welcher W. funktioniert dieses Netzwerk in dem Sinne, dass es einen ununterbrochenen Pfad von $Quelle$ zum $Ziel$ gibt?\\

\begin{alignat*}{3}
&A_i &&: \text{Komponente $i$ funktioniert}, i = 1,\ldots,5 \\
&B   &&: \text{Das Gesamtnetz funktioniert} \\
&B_1 &&: \text{Unterer Pfad passierbar} \\
&B_2 &&: \text{Oberer Pfad passierbar}
\end{alignat*}

Klar:
$
	B_1 = A_4 \cap A_5
$, also
\[
	P(B_1) = P(A_4\cap A_5) = P(A_4)\cdot P(A_5) = p\cdot p = p^2
\]

Auch klar:
$
	B_2 = (A_1 \cap A_2) \cup (A_1 \cap A_3)
$, also
\begin{align*}
	P(B_2) &= P((A_1\cap A_2) \cup (A_1\cap A_3))  \\
               &= P(A_1 \cap A_2) + P(A_1 \cap A_3) - P(A_1\cap A_2\cap A_3) \\
               &= p^2 + p^2 - p^3
\end{align*}

Insgesamt:
\begin{alignat*}{3}
	P(B)            &= \ && P(B_1 \cup B_2) = P(B_1) + P(B_2) - P(B1 \cap B_2) \\
	P(B_1 \cap B_2) &= \ && P\left( A_4 \cap A_5 \cap ((A_1 \cap A_2) \cup (A_1 \cap A_3)) \right) \\
                        &= \ && P(A_4 \cap A_5 \cap A_1 \cap A_2) + P(A_4 \cap A_5 \cap A_1 \cap A_3) - \\
                        &  \ && P(A_1 \cap A_2 \cap A_3 \cap A_4 \cap A_5) \\
                        &= \ && p^4 + p^4 - p^5
\end{alignat*}
Damit:
\[
	P(B) = p^2 + 2p2 - p^3 - (2p^4-p^5) = p^2(3-p-2p^2+p^3)
\]
