\part{Laplace-Experimente, elementare Kombinatorik}

Unser Modell für ein Zufallsexperiment ist ein sogenannter \emph{WRaum}
$(\Omega,\mathcal{A},\mathcal{P})$, wobei $\Omega$ die möglichen
\emph{Ergebnisse} des Experiments enthält, $\mathcal{A}$ ein System von
Teilmengen von $\Omega$ ist, die \emph{Ereignisse} (für uns in der Regel die
Potenzmenge $\mathcal{P}(\Omega)$), und $P$ eine Funktion, die jedem Ereignis
$A$ eine Wahrscheinlichkeit $P(A)$ zuordnet.

Spezialfall Laplace-Experimente:

\begin{displaymath}
\#\Omega < \infty, \ \mathcal{A} = \mathcal{P}(\Omega), \ P(A) =
\frac{\#A}{\#\Omega}
\end{displaymath}

(vielleicht schon bekannt: Wahrscheinlichkeit als Anzahl der günstigen
Ergebnisse durch die Anzahl der möglichen Ergebnisse)

Ergibt sich aus bzw. ist äquivalent zu der Aussage, dass alle
Elementarereignisse dieselbe Wahrscheinlichkeit haben. Typische Anwendung:
Werfen eines symmetrischen Gegenstandes (Würfel, Münze, etc.)


\subsection{Beispiel: Würfelwurf}

Ein Würfel wird zweimal geworfen. Mit welcher Wahrscheinlichkeit erhält man
Augensumme 7 bzw. 6?

Wir betrachten das Laplace-Experiment über
\begin{align*}
\Omega &= \{ (i,j) : 1 \leq i,h \leq 6 \} = \{ 1,\ldots,6 \} \times
\{ 1,\ldots,6 \} \\
\#\Omega &= 36 \\
A_7 &= \{ (i,j) \in \Omega : i+j = 7\} = \{ (1,6), (2,5), (3,4), (4,3), (5,2),
(6,1) \} \\
P(A_7) &= \frac{6}{36} = \frac{1}{6} \\
A_6 &= \{ (1,5), (2,4), (3,3), (4,2), (5,1) \} \\
P(A_6) &= \frac{5}{36} ( < P(A_7))
\end{align*}


\section{Die Kunst des Zählens}

Bei Laplace-Experimenten läuft die Bestimmung von Wahrscheinlichkeiten also
auf die ``Kunst des Zählens'' (elementare Kombinatorik) hinaus. Zwei zentrale
Regeln:

\begin{enumerate}
\item Gibt es eine bijektive Abbildung von $A$ nach $B$, so gilt $\#A = \#B$
\item Sind $A$ und $B$ disjunkt, so gilt $\#(A \cup B) = \#A + \#B$
\end{enumerate}

Regel 2 lässt sich auf mehr als zwei Mengen verallgemeinern: Sind
$A_1,\ldots,A_n$ disjunkt, so gilt $\#(A_1 + \ldots + A_n) = \#A_1 + \#A_2 +
\ldots + \#A_n$.

Wichtige Folgerung: Im Falle $C \subset A \times B$ mit $\#\{ y : (x,y) \in C
\} = k$ fuer alle $x \in A$ gilt $\#C = k \cdot \#A$.


\subsection{Beispiel: Hörsaal}

Hörsaal mit $n$ Reihen mit jeweils $m$ Plätzen. In jeder Reihe sind $k$
Plätze besetzt. Man hat insgesamt $n \cdot k$ Zuhörer.

Spezialfall: Bei $C = A \times B$ erhält man $\#(A \times B) = \#A \cdot \#B$.


\section{Vier Standardfamilien}

Permutationen/Kombinationen mit/ohne Wiederholung.

\subsection{Permutationen mit Wiederholung}

Wieviele Moeglichkeiten gibt es $m$ Objekte (Kugeln) auf $n$ Plätze (Urnen)
zu verteilen? Man hat $n$ Möglichkeiten für das 1. Objekt, wieder $n$ fuer
das 2. etc., also insgesamt $n^m$ Möglichkeiten. Formal:

\[ \{ (i_1,\ldots,i_m) : i_j \in \{ 1,\ldots,n \} \text{ für } j = 1,\ldots,m
\} = n^m \]

Dabei bedeutet das Tupel $(i_1,\ldots,i_m)$, dass Objekt $j$ in Urne $i_j$
gelegt wird. Allgemeiner: Bei endlichen Mengen $A$ und $B$ bezeichnet $B^A$
die Menge aller Funktionen $f: A \to B$, also: $\#(B^A) = (\#B)^{\#A}$.


\subsection{Permutationen ohne Wiederholung}

Was passiert, wenn man nur \emph{injektive} Funktionen zulässt? Klar: Man
braucht $\#B \geq \#A$. Dies setzen wir voraus. Man hat $n$ Möglichkeiten
für das erste Objekt. Ist dieses ausgeteilt, so bleiben $n-1$ Möglichkeiten
fuer das zweite Objekt (Regel 2). Sind die ersten beiden ausgeteilt, so
bleiben $n-2$ für das dritte Objekt etc., man hat also

\[ n \cdot (n-1) \cdot (n-2) \cdot (n-3) \cdot \ldots \cdot (n-m+1) =
\frac{n!}{(n-m)!} = (n)_m \]

Möglichkeiten.
Spezialfall: $A=B$. Dann ist jedes injektive $f$ automatisch surjektiv, also
bijektiv. Allgemein:

\begin{align*}
\#\{ f \in B^A : f \text{ injektiv} \} &= \frac{(\#B)!}{(\#B - \#A)!} \\
A=B : \#\{ \underbrace{f: A \to A : f \text { bijektiv}}_{ \substack{\text {
\scriptsize Die Elemente dieser Menge nennt}\\ \text{\scriptsize man \emph
{Permutationen} von $A$ }}} \} &= (\#A)!
\end{align*}

Standardanwendung: Kartenmischen. Bei $32$ Karten gibt es $32!$ Möglichkeiten
fuer das Mischergebnis.


\subsection{Kombination ohne Wiederholung}

Formal: Wir führen auf $B^A$ eine Äquivalenzrelation ein:

\[ f \sim g :\Leftrightarrow \exists \pi: A \to A \text{ bijektiv}: f = g \circ
\pi \]
\begin{alignat*}{3}
(&f \sim f  \text{ ?}&\quad
&\text{wegen } f = f \circ \pi \text{ mit } \pi = id \\
&f \sim f \Rightarrow g \sim f \text{ ? }&
&f = g \circ \pi \text{ impliziert } g = f \circ \pi^{-1} \\
&f \sim g, g \sim f \Rightarrow f \sim h \text{ ?}&
&f = g \circ \pi, g = h \circ \sigma \Rightarrow f = h \circ \sigma \circ \pi)
\end{alignat*}

Führt auf Zerlegung von $\{ f \in B^A : f \text{ injektiv}\}$ in
Äquivalenzklassen. Wie viele gibt es? Alle haben dieselbe Anzahl von
Elementen, nämlich $m!$ (bei $\#A = m$).


\subsubsection{Beispiel: Lotto}

Beim Lotto 6 aus 49 erhält man zunächst ein Tupel $(i_1,\ldots,i_6)$ mit $i_j
\in \{1,\ldots,49\}$ und $i_j \neq i_k$ fuer $j \neq k$ (eine injektive
Abbildung von $A = \{1,\ldots,6\}$ in $B = \{1,\ldots,49\}$). Dies wird
aufsteigend angeordnet:

\[ k_1,\ldots,k_6 \text{ mit } k_1 < \ldots < k_6 \]

Es gibt $6!$ Möglichkeiten. Formal gilt bei injektiven Funktionen $f : f
\circ \pi = f \circ \sigma \Rightarrow \pi = \sigma$, d.h. die Anzahl der
Elemente einer Äquivalenzklasse ist gleich der Anzahl der Permutationen
(bij: Selbstabbildungen) der Menge $A$, also $(\#A)!$. Also:

\[ \#\{ f \in B^A : f \text{ injektiv} \} = \frac{(\#B)!}{(\#B - \#A)!} \]

Jeweils $(\#A)!$ Elemente werden zu einer Äquivalenzklasse zusammengefasst,
d.h. die gesuchte Anzahl ist

\[ \frac{(\#B)!}{(\#B - \#A)! \cdot (\#A)!} = {\#B \choose \#A} \]

Alternativ:

\[ \{ (i_1,\ldots,i_m): 1 \leq i_1 < i_2 < \ldots < i_m \leq n \} = {n \choose
m} \]

Spezialfall: Wieviele Teilmengen vom Umfang $m$ hat eine Menge von $n$
Elementen? Antwort: $n \choose m$.


\subsection{Kombinationen mit Wiederholung}

\[ \{(i_1,\ldots,i_m): 1 \leq i_1 \leq i_2 \leq \ldots \leq i_m \leq n \} =
{n+m-1 \choose m} \]

\textbf{Beweis mit Regel 1:} Wir können die Menge der $m$-Kombinationen aus
$n$ \emph{mit} Widerholung bijektiv abbilden auf die Menge der
$m$-Kombinationen aus $n+m-1$ \emph{ohne} Wiederholung durch

\[ (i_1,\ldots,i_m) \mapsto (i_1, i_2+1, i_3+2, \ldots, i_m + m-1) \]


\subsection{Zusammenfassende Tabelle}

\begin{center}
\begin{tabular}{l|c|c}
    ~         & Permutationen       & Kombinationen     \\
    \cline{2-3}
    mit Wdh.  & $n^m$               & $n+m-1 \choose m$ \\
    ohne Wdh. & $\frac{n!}{(n-m)!}$ & $n \choose m$     \\
\end{tabular}
\end{center}

{\scriptsize Grundmengen $A,B$ mit $\#A=n$, $\#B=m$.}


