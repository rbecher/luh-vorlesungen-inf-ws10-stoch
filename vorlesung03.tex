\section{Kombinatorik-Beispiele}

\subsection{Beispiel: 10-facher Münzwurf}

Mit welcher W. kommt exakt 7-mal ``Kopf''?
$0$ sei Wappen, $1$ sei Kopf.

Das Modell ist offenbar ein Laplace-Experiment über $\{0,1\}^10$, also ist 
\[\Omega=\{0,1\}^10 = \{ (i_1,\ldots,i_10) : i_j \in \{0,1\}, j = 1,\ldots,10 \}\]

$(1,1,1,1,1,0,0,0,0,0)$ ist das Ereignis/Ergebnis ``Die ersten 5 Würfe ergeben ``Kopf'', alle späteren ``Wappen'' ''

\noindent \textbf{Allgemein gilt}: 
$\omega = (\i_1,\ldots,i_10)$ heißt, dass im $j$-ten Versuch das mit $i_j$ kodierte Ergebnis kommt.\\
\textbf{Nächster Schritt}:
Identifiziere das interessierende Ereignis als Teilmenge $A \subset \Omega$:
\[ A = \{ \omega=(i_1,\ldots,i_10) \in \Omega : i_1+\ldots + i_10 = 7 \} \]
\textbf{In Worten}:
Alle $10$-Tupel, die an $7$ Positionen eine $1$ und an $3$ Positionen eine $0$ haben.

\noindent Wie immer bei Laplace-Experimenten gilt: 
\[P(A) = \frac{|A|}{|\Omega|} \]
Klar ist: $|\Omega| = 2^{10} = 1024$ ($\Omega$ besteht aus den $10$-Permutationen mit Wiederholung aus einer Menge von $2$ Elementen).

\noindent Die nächste Frage heißt: 
Wie viele Elemente hat $A$?\\
Die $7$ $1$en müssen auf die $10$ möglichen Positionen (ohne Wiederholung) verteiltwerden.
$7$-Kombination ohne Wiederholung mit $10$ ELementen:
\[ 10 \choose 7 = \frac{10!}{7!3!} = \frac{10\cdot 9 \cdot 6}{3 \cdot 2} = 5 \cdot 3 \cdot 8 = 120 \]
Also \[ P(A) = \frac{120}{1024} = \frac{15}{128} \]

\subsection{Beispiel: Geburtstagsproblem}

Wie groß ist die W. dafür, dass von $n$ Personen (mindestens) zwei am gleichen Tag Geburtstag haben?
Vernünftige Einschränkungen an $n$: $n \geq 2, n \leq 365$ (Wir ignorieren Schaltjahre, Zwillinge, etc.)

Wie gehen wieder von einem Laplace-Experiment aus, und zwar über
\[ \Omega = \{ (i_1,\ldots,i_n) : i_j = \{ 1,\ldots,365 \} \} \]
Dabei bedeutet $i_j=t$, dass die $j$-te Person am $k$-ten Tag des Jahres Geburtstag hat.
$|\Omega| = 365^n$ ($n$ Permutationen mit Wiederholung aus einer Menge mit $365$ Elementen.

\textbf{Nächster Schritt} (wieder):
Identifiziere $A$ als Teilmenge von $\Omega$.
\[ A = \{ (i_1,\ldots,i_n) \in \Omega : \exists j_k=\{1,\ldots,n\}\text{ mit }j\neq k \cap i_j=i_k  \} \]

Hier nützlich: Übergang zum Komplementär-Ereignis
\[ A^c = \{ (i_1,\ldots,i_n) \in \Omega : i_j \neq i_k,\ j \neq k \}\]

$A^c$ ist also die Menge der $n$-Permutationen aus einer Menge von $365$ Elementen, nun also \textbf{ohne} Wiederholung:
$365 \cdot 364 \cdot \ldots \cdot (365 - n +1)$
Also
\[ P(A) = 1-P(A^c) = 1 - \frac{365 \cdot 364 \cdot \ldots \cdot (365 - n +1)}{365^n} \]

Man sieht an der Formel, dass $P(A)$ mitwachsenden $n$ größer wird.
Interessanterweise erhält man ab $n=23$ einen Wert $P(A) \geq \frac{1}{2}$.

\subsection{Beispiel: Wichteln}

$n$ Kinder verpacken jeweils ein Geschenk.
Diese werden zufällig an die Kinder verteilt: jedes Kind erhält also exakt ein Geschenk.

Wie groß ist nun die Wahrscheinlichkeit, dass dies ``ohne Tränen funktioniert'', so dass kein Kind sein eigenes Geschenk erhält?
$\Omega$ sei die Menge aller $n$-Permutationen ohne Wiederholung aus $n$ Interpretationen.
$\omega = (i_1,\ldots,i_n)$ mit $i_j=k$ bedeutet, dass dsa $j$-te Kind Geschenk mit der Nummer $k$ erhält.
Also $|\Omega| = n!$

$A$ ist nun die Menge der fixpunktfreien Permutationen. Also ist $A^c$ die Menge aller Permutationen, die mindestens einen Fixpunkt haben.

Wir suchen
\[ B_j = \{ (i_1,\ldots,i_n)\in \Omega:\ i_j=j \} \]
($j$ ist ein Fixpunkt, Kind Nummer $j$ erhält sein eigenes Geschenk zurück)

Klar: \[ A^c = \bigcup_{j=1}^n B_j \]

Nebenrechnung: $|B_j = (n-1)!$

Schade: Die $B_j$ sind nicht disjunkt.


\[ A^c = \{ (i_1,\ldots,i_n) \in \Omega:\ \exists j \in \{1,\ldots, n\} \text{ mit } i_j=j \]

Typisches Anwendungsgebiet fürdie Siebformel: % TODO Href dorthin
Für $n=2$ gilt also z.B.
\[ P(B_1 \cup B_2) = P(B_1) + P(B_2) - P(B_1\cap B_2) \]

Bei $|H| = k$ besteht $\bigcap_{j\in H}B_j$ aus allen Permutationen, deren Werte in den $k$ Positionen aus $H$ festgelegt sind, die übrigen $n-k$ sind frei.

Für alle $H$ mit $|H|=k$ erhält man also $P\left(\bigcap_{j\in H} B_j\right) = \frac{(n-k)!}{n!}$.
Hängt also nur von $|H|$ ab.

Wie vele $H \subset \{ 1,\ldots,n \}$ mit $|H|=k$ gibt es?

$k$-Kombinationen ohne Wdhl. aus $|H|$ ist $n \choose k$

Damit 
\[P(A) = 1- P(A^c) = 1-P\left(\bigcup_{j=1}^n B_j\right) = 1-\sum_{k=1}^n(-1)^{k+1} \cdot {n \choose k} \cdot \frac{(n-k)!}{n!} = \sum_{k=0}^n \frac{(-1)^k}{k!}\] % TODO Sieht seltsam aus ...

Aus der Analysis ist die Reihendarstellung der Experimentalfunktion bekannt
\[e^x = \sum_{k=0}^\infty \frac{x^k}{k!}\]

Mit $n\rightarrow\infty$ geht die gesuchte Wahrscheinlichkeit also gegen $e^{-1} = \frac{1}{e} \approx 0.3679\ldots$

\subsection{Beispiel: Kartenspiel}

Ein Kartenspiel mit $52$ Karten wird gut gemischt, die oberen $5$ werden umgedreht.
Mit welcher Wahrscheinlichkeit erhält man Ereignis $A$, einen ``Flush'' (alle haben die gleichen Farben), bzw. Ereignis $B$, $4$ Karten mit der gleichen Wertigkeit?

\[P(A) = 4 \cdot \frac{{13 \choose 5}}{{52 \choose 5}} \] (Möglichkeiten für die Farbe mal die Möglichkeit, aus der Farbe 5 auszusuchen durch Gesamtmenge)

\[ P(B) = \frac{13 \cdot 48}{{52 \choose 5}} \]
(Auswahl des Kartenwerts mal Möglichkeit für die 5. karte(???) durch Gesamtmenge)

\[ \frac{P(A)}{P(B)} = \frac{33}{4} \]

Also ....