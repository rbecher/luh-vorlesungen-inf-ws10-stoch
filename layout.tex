%% Basierend auf einer TeXnicCenter-Vorlage von Tino Weinkauf.

%%%%%%%%%%%%%%%%%%%%%%%%%%%%%%%%%%%%%%%%%%%%%%%%%%%%%%%%%%%%%%


%%%%%%%%%%%%%%%%%%%%%%%%%%%%%%%%%%%%%%%%%%%%%%%%%%%%%%%%%%%%%

%% OPTIONEN

%%%%%%%%%%%%%%%%%%%%%%%%%%%%%%%%%%%%%%%%%%%%%%%%%%%%%%%%%%%%%

%%

%% ACHTUNG: Sie benötigen ein Hauptdokument, um diese Datei

%%          benutzen zu können. Verwenden Sie im Hauptdokument

%%          den Befehl "\input{dateiname}", um diese

%%          Datei einzubinden.

%%


%%%%%%%%%%%%%%%%%%%%%%%%%%%%%%%%%%%%%%%%%%%%%%%%%%%%%%%%%%%%%

%% OPTIONEN FÜR ABSTÄNDE

%%%%%%%%%%%%%%%%%%%%%%%%%%%%%%%%%%%%%%%%%%%%%%%%%%%%%%%%%%%%%


%%Abstand zwischen den Absätzen: halbe Höhe vom kleinen x

\setlength{\parskip}{0.5ex}


%%Einzug am Anfang eines Absatzes: auf Null setzen

%\setlength{\parindent}{0ex}


%%Zeilenabstand: 1.5 fach

%% ==> Erwägen Sie, anstelle dieses Kommandos das Paket 'setspace' zu verwenden.

%\linespread{1.5}



%%%%%%%%%%%%%%%%%%%%%%%%%%%%%%%%%%%%%%%%%%%%%%%%%%%%%%%%%%%%%

%% OPTIONEN FÜR KOPF- UND FUSSZEILEN

%%%%%%%%%%%%%%%%%%%%%%%%%%%%%%%%%%%%%%%%%%%%%%%%%%%%%%%%%%%%%

%%Beispiel für recht nette Kopf- und Fußzeilen

%% ==> Nutzen Sie '\usepackage{fancyhdr}' und '\pagestyle{fancy}'

%% ==> im Hauptdokument, um diese zu benutzen.

%\pagestyle{fancy}

\renewcommand{\chaptermark}[1]{\markboth{#1}{}}

\renewcommand{\sectionmark}[1]{\markright{\thesection\ #1}}

\fancyhf{}

\fancyhead[LE,RO]{\thepage}

\fancyhead[LO]{\rightmark}

\fancyhead[RE]{\leftmark}

\fancypagestyle{plain}{%

    \fancyhead{}

    \renewcommand{\headrulewidth}{0pt}

}



