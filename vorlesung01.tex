\part{Einf\"uhrung}

\section{Inhalte}

Dies ist ein Paragraph

\newpage

\part{Inhalt}

\section{Zufall}

Ein weiterer Absatz mit etwas Inhalt .... und einer bekannten Formel!
\begin{displaymath}
\sum_{i=0}^{n} i = \frac{n(n+1)}{2}
\end{displaymath}

In Formel \ref{kleinergauss} steht der bekannte Satz nochmal, aber diesmal k\"onnen wir uns darauf direkt beziehen. Unverlinkt sieht das so aus: \ref*{kleinergauss}. 

\begin{equation}
\label{kleinergauss}
\sum_{i=0}^{n} i = \frac{n(n+1)}{2}
\end{equation}

Die Formel geht \"ubrigens auch inline ($\sum_{i=0}^{n} i = \frac{n(n+1)}{2}$), sieht dann aber nicht so h\"ubsch aus \ldots \\
Auch wichtig sind so genannte ``equation arrays''.

\begin{eqnarray}
A & = & B \\
C & = & D 
\end{eqnarray}