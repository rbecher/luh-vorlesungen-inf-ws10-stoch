%\part{Einf\"uhrung}

%\section{Inhalte}

%Dies ist ein Paragraph

%\newpage

\part{Ein mathematisches Modell für Zufallsexperimente}

Man fasst die möglichen Ergebnisse $\omega$ zu einer Menge $\Omega$ zusammen; man nennt $\Omega$ den \textit{Ergebnisraum} oder auch \textit{Stichprobenraum}. \textit{Ereignisse} werden durch Teilmengen von $\Omega$ beschrieben.
Eine \textit{Aussage} über das Ergebnis wird zu der Menge $A$ aller $\omega \in \Omega$, für diese die Aussage richtig ist.

\section{Beispiel - Würfelwurf}

\begin{displaymath}
\Omega = \{1,2,3,4,5,6\}
\end{displaymath}

Das Ereignis ``Es kommt eine gerade Zahl heraus'' wird beschrieben durch

\begin{displaymath}
A = \{2,4,6\}
\end{displaymath}

Das Ereignis ``Es kommt eine 6 heraus'' wird (in Mengenschreibweise) beschrieben durch $A = \{6\}$.
Solche Ereignisse, die nur aus einem Element bestehen, nennt man \textit{Elementarereignisse}.
Aussagen über das Ergebnis können \textit{logisch kombiniert} werden. Auf der Ereignisseite (also in Teilmengen von $\Omega$) wird dies zu Mengenoperationen. Werden zwei Ereignisse $A, B$ durch Mengen repräsentiert, so ist

\begin{align*}
A^c &:\text{ das Ereignis, dass $A$ nicht eintritt}\\
A \cap B &: \text{ das Ereignis, dass $A$ \textit{und} $B$ eintreten}\\
A \cup B &: \text{ das Ereignis, dass (mindestens) eines der Ereignisse $A,B$ eintritt}
\end{align*}

Hier kommt eine Grafik über eine Matrix .... oder ein Foto der Tafel
A = Pasch\\
B = Augensumme kleiner 5\\
A geschnitten B

Man kann dies mit mehr als zwei Ereignissen machen.
Wenn $A,B,C$ Ereignisse sind, wodurch wird ''genau eins davon tritt ein'' beschrieben?

\begin{equation*}
A \cap B^c \cap C^c + A^c \cap  B \cap C^c + A^c \cap B^c \cap C
\end{equation*}
% TODO entweder die Eqn doppeln oder auf das + hinweisen wegen disjunkter Mengen

\section{Was ist Wahrscheinlichkeit?}

Unsere zentrale Frage lautet nun: ``Was ist Wahrscheinlichkeit?''\\
Der Mathematiker wählt ein Axiomensystem (vgl. Geometrie/Raum).
Ausgangspunkt hierfür ist der Alltagsbegriff.
Was bedeutet es also, wenn beim Wurf einer fairen Münze mit Wahrscheinlichkeit $\frac{1}{2}$ das Ergebnis ``Kopf'' kommt?\\

\subsection{Frequentistische Auffassung}

Bezeichnet $N_n(A)$ die Anzahl der Versuche bei $n$ Wiederholungen des Experiments, bei dem $A$ kommt, so sollte die \textbf{relative Häufigkeit} $\frac{1}{n} N_n(A)$ bei großem $n$ in der Nähe der Wahrscheinlichkeit $p$ des Ereignisss $A$ liegen.

\subsection{Subjektivistische Auffassung}

$p$ drückt den ``Grad meines Glaubens'' an das Eintreten von $A$ auf einer Skala von $0$ (``kein Glaube'') bis $1$ (``Gewissheit'') aus.
Kann über Wetten formalisiert werden.

\noindent \textbf{Anmerkung:} Diese beiden Auffassungen sind nicht disjunkt!

Für relative Häufigkeiten gilt
\begin{align*}
0 \leq \frac{1}{n} N_n(A) & \leq 1 \\
\frac{1}{n} N_n(\Omega) & = 1 \\
\frac{1}{n} N_n(\emptyset) & = 0\\
\frac{1}{n} N_n(A+B) & = \frac{1}{n} N_n(A) + \frac{1}{n} N_n(B)
\end{align*}

\section{Definition: Kolmogorov-Axiome}

Gegeben seien ein (nicht leerer) Ergebnisraum $\Omega$ und ein System $\mathcal{A}$ von Teilmengen von $\Omega$, das System der Ereignisse (bei endlichen oder abzählbar unendlichen $\Omega$ ist $\mathcal{A}$ einfach die Potenzmenge $\mathcal{P}(\Omega)$, also die Menge \textbf{aller} Teilmengen von $\Omega$).
Eine \textbf{Wahrscheinlichkeit} oder besser ein \textbf{Wahrscheinlichkeitsmaß} (WMaß) ist eine Abbildung $P:\mathcal{A} \rightarrow \mathbb{R}$ mit folgenden Eigenschaften:
\begin{description}
 \item[(A1)] $P(A)\geq 0\ \forall A \in \mathcal{A},\ P(\Omega)=1$ ($P$ steht für ``probability'')
 \item[(A2)] Für alle paarweise disjunkten Ereignisse $A_1,A_2,A_3,\ldots$ gilt
 \begin{displaymath}
	P(\bigcup_{i=1}^\infty A_i) = P(\sum_{i=1}^\infty A_i) = \sum_{i=1}^\infty P(A_i)
 \end{displaymath}
 Dies nennt man ``$\sigma$-Additivität''.
\end{description}

\section{Wahrscheinlichkeitsraum}
Das Tripel $(\Omega,\mathcal{A},P)$ nennt man einen \textbf{Wahrscheinlichkeitsraum}.

\noindent Es folgen nun erste Folgerungen aus den Axiomen.

\section{Satz}
% TODO: Enumerate soll mit kleinen Buchstaben arbeiten
\renewcommand{\labelitemi}{\alph}
\begin{enumerate}
	\item $P(\emptyset) = 0$
	\item $P(A) \leq 1 \ \forall A \in \mathcal{A}$
	\item $P(A^c) = 1 - P(A) \ \forall A \in \mathcal{A}$ (Übergang zum \textbf{Gegenereignis})
	\item $A \subset B \Rightarrow P(A) \leq P(B)$ (Monotonie)
	\item $P(A_1 + A_2 + \ldots + A_n) = P(A_1) + P(A_2) + \ldots + P(A_n)$ für paarweise disjunkte $A_1,\ldots,A_n \in \mathcal{A}$ (endliche Additivität)
	\item Boolesche Ungleichung: $P(A_1 \cup \ldots \cup A_n) \leq P(A_1) + \ldots + P(A_n)$
	\item Die \textbf{Siebformel}
	\[
		\mathsf{P}\left(\bigcup_{i=1}^nA_i\right) = \sum_{k=1}^n(-1)^{k+1}\!\!\sum_{I\subseteq\{1,\dots,n\},\atop |I|=k}\!\!\!\!\mathsf{P}\left(\bigcap_{i\in I}A_i\right)
	\]
\end{enumerate}

\subsection{Beweis}
\begin{enumerate}
	\item[1.] Mit (A2) und $A_j=\emptyset \ \forall j$ erhält man $P(\emptyset) = \sum_{i=1}^\infty P(\emptyset)$. 
	Wegen $P(\emptyset) \in \mathbb{R}$ bedeutet dies $P(\emptyset) = 0$.
	\item[5.] Verwende (A2) mit $A_j=\emptyset$ für $j>k$ und dann Teil (a):
	\begin{align*}
		P(A_1 \cup \ldots \cup A_k) & =
		P(A_1 \cup \ldots \cup A_k \cup \emptyset \cup \emptyset \ldots) \\ & = 
		P(A_1) + \ldots + P(A_k) + P(\emptyset) + \ldots \\ & =
		P(A_1) + \ldots + P(A_k)
	\end{align*}
	\item[3.] $1 = P(\Omega) = P(A + A^c) = P(A) + P(A^c)$
	\item[4.] 
	\begin{alignat*}{3}
	    &B    && = A    && + B \cap A^c\\
	    &P(B) && = P(A) && + \underbrace{P(B\cap A^c)}_{\geq 0 \ \text{nach (A1)}} \geq P(A)
	\end{alignat*}
	\item[7.] bei $n=2$: $P(A \cup B) \neq P(A) + P(B)$
\end{enumerate}


\textbf{Hinweis:} Jetzt seid ihr so weit in dieser Mitschrift gekommen, bitte gleicht diese mit eurer handschriftlichen Mitschrift ab und sendet mir eure Verbesserungen. Gerne auch gleich in \LaTeX\ ;-)
